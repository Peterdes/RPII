\documentclass[12pt]{article}
\usepackage{polski}
\usepackage[utf8]{inputenc}
\usepackage[T1]{fontenc}
\usepackage{amsmath}
\usepackage{amsfonts}
\usepackage{fancyhdr}
\usepackage{lastpage}
\usepackage{multirow}
\usepackage{amssymb}
\usepackage{amsthm}
\usepackage{textcomp}
\usepackage{bbm}
\frenchspacing
\usepackage{fullpage}
\setlength{\headsep}{30pt}
\setlength{\headheight}{12pt}
%\setlength{\voffset}{-30pt}
%\setlength{\textheight}{730pt}
\pagestyle{myheadings}
%\usepackage{kuvio,amscd,diagrams,dcpic,xymatrix,diagxy}
\usepackage{tikz,paralist,mathtools}

\newcommand{\M}{\mathfrak{M}}
\newcommand{\F}{\mathcal{F}}
\newcommand{\G}{\mathcal{G}}
\renewcommand{\P}{\mathbb{P}}
\newcommand{\E}{\mathbb{E}}
\newcommand{\R}{\mathbb{R}}
\newcommand{\B}{\mathcal{B}}
\newcommand{\N}{\mathcal{N}}
\newcommand{\1}{\mathbbm{1}}
\newcommand{\var}{\mathrm{Var}}
\newcommand{\cov}{\mathrm{Cov}}
\newcommand{\corr}{\mathrm{Corr}}
\newcommand{\pn}{\mathrm{p.n.}}
\newcommand{\sgn}{{\mathrm{sgn}}\,}
\newcommand{\bigslant}[2]{{\left.\raisebox{.2em}{$#1$}\middle/\raisebox{-.2em}{$#2$}\right.}}
\newcommand{\mf}[1]{{\mathfrak{#1}}}
\newcommand{\mb}[1]{{\mathbb{#1}}}
\newcommand{\mc}[1]{{\mathcal{#1}}}
\newcommand{\mr}[1]{{\mathrm{#1}}}
\newcommand{\ds}{\displaystyle}
\newcommand{\res}{{\mathrm{res}}}

\newcounter{punkt}

\theoremstyle{plain}
\newtheorem{theorem}[punkt]{Theorem}
\newtheorem{theoremnp}[punkt]{Theorem (no proof)}
\newtheorem{lemma}[punkt]{Lemma}
\newtheorem{proposition}[punkt]{Proposition}

\theoremstyle{definition}
\newtheorem{definition}[punkt]{Definition}
\newtheorem{fact}[punkt]{Fact}
\newtheorem{corollary}[punkt]{Corollary}

\theoremstyle{remark}
\newtheorem{remark}[punkt]{Remark}
\newtheorem{example}[punkt]{Example}
\newtheorem{exercise}[punkt]{Exercise}

\markright{Piotr Suwara\hfill Rachunek prawdopodobieństwa II: 13 listopada 2013 \hfill}

\begin{document}
	\begin{theorem}[CLT Lindeberg-Levy]
		Let $(X_{n,k})$ be a triangular array,
		$\sum_{k=1}^{k_n} X_{n,k} \to m$,
		$\sum_{k=1}^{k_n} \var(X_{n,k}) \to \sigma^2$,
		and Lindeberg's condition be satisfied:
		$$ \forall_{\varepsilon>0} \sum_{k=1}^{k_n}
			\E |X_{n,k} - \E X_{n,k} |^2 
			\1_{\{ |X_{n,k} - \E X_{n,k} | > \varepsilon \}} \to 0.$$
		Then $$S_n = \sum_{k=1}^{k_n} X_{n,k} \implies \N(m, \sigma^2).$$
	\end{theorem}
	
	\begin{proposition}
		For $X_1, X_2, \ldots$ iid, $\var(X_1)<\infty$,
		array $\displaystyle X_{n,k} = \frac{X_k - \E X_k}{\sqrt{n}}$
		satisfies Lindeberg's condition.
	\end{proposition}
	
	\begin{proposition}[Lyapunov condition]
		Lyapunov's condition
		$$ \exists_{\delta>0} \sum_{k=1}^{k_n}
			\E |X_{n,k} - \E X_{n,k} |^{2 + \delta} \to 0 $$
		implies Lindeberg's condition (under CLT's assumptions
		about $\E$ and $\var$ of sums).
	\end{proposition}
	
	\begin{corollary}
		For $\sum \E X_{n,k} \to 0, \sum \var(X_{n,k}) \to 1$,
		it follows that $$ \sup_t \left| \P\left(\sum X_{n,k} \leq t\right)
			- \phi(t)\right| \to 0,$$
		and in particular $\P(\sum X_{n,k} \leq t) \to \phi(t)
		= \P(\N(0,1) \leq t)$.
	\end{corollary}
	
	\begin{theorem}[Berny-Esseen]
		$X_1, X_2, \ldots, X_n$ independent,
		$\E X_i = 0, \sum_{k=1}^n \var(X_k) = 1$, then
		$$\left| \P\left(\sum_{k=1}^n X_k \leq t\right) - \phi(t) \right|
		\leq C_1 \sum_{k=1}^n \E |X_k|^3.$$
	\end{theorem}
	
	\begin{corollary}
		In particular, for $X_1, \ldots, X_n$ iid, $\E X_1 = m$,
		$\var(X_1) = \sigma^2$, we have
		$$\left| \P\left(\frac{\sum_{k=1}^n X_k - nm}{\sqrt{n} \sigma} \leq t \right)
				- \phi(t) \right|
		\leq C_2 \sum_{k=1}^n \E \left| \frac{X_k - \E X_k}{\sqrt{n}\sigma}\right|^3
		= C_2 \frac{\E |X_1 - \E X_1|^3}{\sqrt{n} \sigma^3}.$$
	\end{corollary}
	
	\begin{remark}
		Obviously $C_2 \leq C_1$, it is known that
		$C_1, C_2 \geq \frac{\sqrt{10}+e}{6 \sqrt{2\pi}} \approx 0.4097$,
		and also $C_1 \leq 0.56, C_2 \leq 0.4748$.
	\end{remark}
	
	\begin{example}[de Moivre-Laplace]
		$S_n \sim \mr{Bin}(n,p)$,
		$\left|\P(S_n \leq t) - \phi\left(\frac{t - np}{\sqrt{n p (1-p)}} \right) 
		\right| \leq C_2 \frac{p^2 + (1-p)^2}{\sqrt{np(1-p)}}$.
	\end{example}
	
	\begin{definition}
		$\G \subset \F$ a~$\sigma$-field, $X$ random variable, $\E |X|<\infty$,
		then $Z=\E(X|\G)$ is a~random variable such that
		it is $\G$-measurable, $\forall_{A \in \G} \E(Z \1_A) = \E(X \1_A)$.
	\end{definition}
	
	\begin{proposition}
		The above exists and is unique up to a~set of probability 0.
	\end{proposition}

	
	\begin{proposition}
		$Y$ random variable, $X$ measurable with respect to $\sigma(Y)$,
		then $X=h(Y)$ for some Borel function $h$.
	\end{proposition}
	
	\begin{definition}
		$\E(X|Y) = \E(X | \sigma(Y))$ for $X, Y$ random variables, $\E |X|<\infty$.
	\end{definition}



	
	
\end{document}
 
 
