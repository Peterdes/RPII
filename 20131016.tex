\documentclass[12pt]{article}
\usepackage{polski}
\usepackage[utf8]{inputenc}
\usepackage[T1]{fontenc}
\usepackage{amsmath}
\usepackage{amsfonts}
\usepackage{fancyhdr}
\usepackage{lastpage}
\usepackage{multirow}
\usepackage{amssymb}
\usepackage{amsthm}
\usepackage{textcomp}
\usepackage{bbm}
\frenchspacing
\usepackage{fullpage}
\setlength{\headsep}{30pt}
\setlength{\headheight}{12pt}
%\setlength{\voffset}{-30pt}
%\setlength{\textheight}{730pt}
\pagestyle{myheadings}
%\usepackage{kuvio,amscd,diagrams,dcpic,xymatrix,diagxy}
\usepackage{tikz,paralist,mathtools}

\newcommand{\M}{\mathfrak{M}}
\newcommand{\F}{\mathcal{F}}
\newcommand{\G}{\mathcal{G}}
\renewcommand{\P}{\mathbb{P}}
\newcommand{\E}{\mathbb{E}}
\newcommand{\R}{\mathbb{R}}
\newcommand{\B}{\mathcal{B}}
\newcommand{\N}{\mathcal{N}}
\newcommand{\1}{\mathbbm{1}}
\newcommand{\var}{\mathrm{Var}}
\newcommand{\cov}{\mathrm{Cov}}
\newcommand{\corr}{\mathrm{Corr}}
\newcommand{\pn}{\mathrm{p.n.}}
\newcommand{\sgn}{{\mathrm{sgn}}\,}
\newcommand{\bigslant}[2]{{\left.\raisebox{.2em}{$#1$}\middle/\raisebox{-.2em}{$#2$}\right.}}
\newcommand{\mf}[1]{{\mathfrak{#1}}}
\newcommand{\mb}[1]{{\mathbb{#1}}}
\newcommand{\mc}[1]{{\mathcal{#1}}}
\newcommand{\mr}[1]{{\mathrm{#1}}}
\newcommand{\ds}{\displaystyle}
\newcommand{\res}{{\mathrm{res}}}

\newcounter{punkt}

\theoremstyle{plain}
\newtheorem{theorem}[punkt]{Theorem}
\newtheorem{theoremnp}[punkt]{Theorem (no proof)}
\newtheorem{lemma}[punkt]{Lemma}
\newtheorem{proposition}[punkt]{Proposition}

\theoremstyle{definition}
\newtheorem{definition}[punkt]{Definition}
\newtheorem{fact}[punkt]{Fact}
\newtheorem{corollary}[punkt]{Corollary}

\theoremstyle{remark}
\newtheorem{remark}[punkt]{Remark}
\newtheorem{example}[punkt]{Example}
\newtheorem{exercise}[punkt]{Exercise}

\markright{Piotr Suwara\hfill Rachunek prawdopodobieństwa II: 16 października 2013 \hfill}

\begin{document}
	\begin{definition}[characteristic function]
		$\displaystyle\varphi_\mu(t) = \int_{\R^d} e^{i\langle t,x\rangle}d\mu(x)$
		
		$\displaystyle\varphi_X(t) = \varphi_{\mu_X}(t) = \E e^{i\langle t,x\rangle}$
	\end{definition}
	
	\begin{proposition}
		$\varphi_X(0) = 1$ and $|\varphi_X(t)| \leq 1, t \in \R^d$
	\end{proposition}
	
	\begin{proposition}
		$\varphi_X$ uniformly continuous on $\R^d$.
	\end{proposition}
	
	\begin{proposition}
		$\varphi_X$ is a~nonnegatively determined function on $\R^d$.
	\end{proposition}
	
	\begin{theorem}[Bochner]
		A~function $\varphi:\R^d \to \mb{C}$ is a~characteristic function of a~random $d$-dimensional vector iff $\varphi(0)=1$, $\varphi$ is continuous and $\varphi$ is nonnegatively determined, i.e. $(\varphi(t_i - t_j))_{i,j}$ is nonnegatively determined for any $t_i$'s.
	\end{theorem}
	
	\begin{proposition}
		$\varphi_{AX+b}(t) = e^{i \langle b,t\rangle} \varphi_X(A^T t)$, in particular $\varphi_{-X}(t) = \varphi_X(-t) = \overline{\varphi_X(t)}$.
	\end{proposition}
	
	\begin{proposition}
		$X$ real random variable, if $\E |X|^k < \infty$, $k \in \mb{Z}_+$, then $\varphi_X \in C^k(\R)$ and $\varphi_X^{(k)}(t) = i^k \E X^k e^{itX}$.
	\end{proposition}
	
	\begin{remark}
		Existence of $\varphi_X'$ does not imply $\E |X| < \infty$.
	\end{remark}
	
	\begin{proposition}
		If $\mu, \nu$ probability measures on $\R^d$ such that $\varphi_\mu = \varphi_\nu$, then $\mu = \nu$.
	\end{proposition}
	
	\begin{proposition}
		$X_1, \ldots, X_n$ independent random $d$-dimensional variables, then $\varphi_{X_1 + \ldots + X_n} (t) = \varphi_{X_1}(t) \ldots \varphi_{X_n}(t)$.
	\end{proposition}








\end{document}
 
 
