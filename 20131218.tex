\input{includepl.tex}

\markright{Piotr Suwara\hfill Rachunek prawdopodobieństwa II: 18 grudnia 2013 \hfill}

\begin{document}
	\begin{theorem}[zbieżność w~$L^1$]
		$(M_n, \F_n)$ martyngał, NWSR:
		\begin{enumerate}[a)]
			\item $\{ M_n \}_{n \geq 0}$ jednostajnie całkowalna,
			\item $M_n$ zbieżny w~$L^1$
			(czyli $\exists_M \E |M_n - M| \to 0$),
			\item $M_n$ jest prawostronnie domknięty
			(czyli $\exists_M$ $M$ całkowalne, 
			$M_n = \E(M | \F_n)$),
			\item[c')] $\exists_{M_\infty}$ 
			$\F_\infty=\sigma\left( \bigcup_{n=1}^\infty \F_n \right)
			$-mierzalna, 
			$M_n = \E (M_\infty | \F_n)$ p.n..
		\end{enumerate}
		
		Ponadto wtedy $M_n \to M_\infty$ p.n. i~w~$L^1$.

	\end{theorem}
	
	\begin{theorem}[zbieżność w~$L^p$]
		$(M_n, \F_n)$ martyngał, $p>1$, $\forall_n \E |M_n|^p < \infty$,
		NWSR:
		\begin{enumerate}[a)]
			\item $\sup_n \E |M_n|^p < \infty$,
			\item $\{|M_n|^p\}_n$ jednostajnie całkowalna,
			\item $M_n$ zbieżny w~$L^p$ 
			(czyli $\exists_M \E |M|^p < \infty, \E |M_n - M|^p \to 0$),
			\item $M_n$ jest prawostronnie domknięty
			przez zmienną z~$L^p$
			\\ (czyli $\exists_M \E |M|^p < \infty, 
			\forall_n \E(M | \F_n) = M_n$),
			\item[d')] $\exists_M \F_\infty$-mierzalna, 
			$\E |M_\infty| < \infty,
			\forall_n \E (M_\infty | \F_n) = M_n$.

		\end{enumerate}
		
		Ponadto wtedy $M_n \to M_\infty$ p.n. i~w~$L^p$.

	\end{theorem}
	
	\begin{remark}
		Istnieje martyngał jednostajnie całkowalny $(M_n)_{n \geq 0}$ 
		taki, że $\sup_n \E |M_n| = \infty$.
	\end{remark}
	
	\pagebreak
	
	{\bf Łańcuchy Markowa}
	
	$E$ -- skończona lub przeliczalna przestrzeń stanów.
	
	\begin{definition}[łańcuch Markowa]
		Proces $(X_n)_{n \geq 0}$ o~wartościach w~skończonej 
		lub przeliczalnej przestrzeni $E$ nazywamy
		\emph{łańcuchem Markowa},
		jeśli zachodzi warunek
		$\P(X_{n+1}=a_{n+1} | X_n = a_n, \ldots, X_0 = a_0)
		= \P(X_{n+1} = a_{n+1} | X_n = a_n)$
		o~ile $\P(X_n = a_n, \ldots, X_0 = a_0) > 0$.
	\end{definition}
	
	\begin{example}
		$X_0, X_1, X_2, \ldots$ niezależne, to $(X_n)$ ł.M.
	\end{example}
	
	\begin{example}
		$X_0, X_1, \ldots$ niezależne, to $(S_n = X_n + S_{n-1})$ ł.M.
	\end{example}
	
	\begin{example}
		Błądzenie po wierzchołkach.
	\end{example}
	
	\begin{definition}[macierz przejścia]
		$X_n$ jest ł.M., \emph{macierzą przejścia} w~$n$-tym kroku
		$(P_n(a,b))_{a,b \in E}$ nazywamy macierz elementów
		$P_n(a,b) = \P(X_n = b | X_{n-1} = a)$ 
		o~ile $\P(X_{n-1} = a) > 0$.
	\end{definition}
	
	\begin{remark}
		$P_n$ macierz przejścia w~$n$-tym kroku, $\P(X_{n-1}=a)>0$,
		wtedy
		\begin{itemize}
			\item $\forall_b P_n(a,b) \geq 0$,
			\item $\sum_b P_n(a,b) = 1$.
		\end{itemize}

	\end{remark}
	
	\begin{definition}[macierz stochastyczna]
		Macierz $P = (p(a,b))_{a,b \in E}$ nazywamy {\em stochastyczną},
		jeśli $\forall_{a,b} p(a,b) \geq 0 $ 
		oraz $\forall_a \sum_b p(a,b) = 1$.
	\end{definition}
	
	\begin{definition}[jednorodny ł. M.]
		Łańcuch Markowa $(X_n)$ nazywamy \emph{jednorodnym} 
		z~macierzą przejścia $P = (p(a,b))$, jeśli
		$\forall_{n,a,b} \P(X_n = b| X_{n-1} = a) = p(a,b)$
		o~ile $\P(X_{n-1} = a) > 0$.
	\end{definition}
	
	\begin{example}
		$X_0, X_1, \ldots$ niezależne, $(X_n)$ jednorodny ł.M.
		wtw, gdy $X_n$ mają jednakowy rozkład.
	\end{example}
	
	\begin{example}
		$X_i$ niezalezne, $(S_n = X_0 + \ldots + X_n)$ jednorodny 
		wtw, gdy $X_n$ mają jednakowy rozkład.
	\end{example}
	
	\begin{example}
		Błądzenie losowe po trójkące, błądzenie losowe na 
		$\{-a, \ldots, b\}$ z~odbiciem lub pochłanianiem
		są jednorodne.
	\end{example}
	
	\begin{definition}[rozkład początkowy]
		\emph{Rozkładem początkowym} ł.M. $(X_n)_{n \geq 0}$ nazywamy
		rozkład $X_0$, czyli ciąg $(\pi_a)_{a \in E}$
		taki, że $\P(X_0 = a) = \pi_a$.
	\end{definition}
	
	\begin{fact}
		$P = (p(a,b))$ macierz stochastyczna, $\Pi = (\pi_a)$ 
		rozkład na $E$.
		$(X_n)$ jest (jednorodnym) ł.M. o~macierzy przejścia $P$ 
		i~rozkładzie poczatkowym $\Pi$ wtw, gdy
		\\ $\P(X_0 = a_0, \ldots, X_n = a_n)
			= \pi_{a_0} p(a_0, a_1) \cdot \ldots \cdot p(a_{n-1}, a_n)$.
	\end{fact}
	
	\begin{theorem}[o~istnieniu ł.M.]
		$\Pi = (\pi_a)_{a \in E}$ dowolny rozkład na $E$,
		$P = (p(a,b))_{a,b \in E}$ macierz stochastyczna na $E$,
		wówczas istnieje (jednorodny) ł.M. 
		o~rozkładzie początkowym $\Pi$
		i~macierzy przejścia $P$.
	\end{theorem}







\end{document}
 
 
