\documentclass[12pt]{article}
\usepackage{polski}
\usepackage[utf8]{inputenc}
\usepackage[T1]{fontenc}
\usepackage{amsmath}
\usepackage{amsfonts}
\usepackage{fancyhdr}
\usepackage{lastpage}
\usepackage{multirow}
\usepackage{amssymb}
\usepackage{amsthm}
\usepackage{textcomp}
\usepackage{bbm}
\frenchspacing
\usepackage{fullpage}
\setlength{\headsep}{30pt}
\setlength{\headheight}{12pt}
%\setlength{\voffset}{-30pt}
%\setlength{\textheight}{730pt}
\pagestyle{myheadings}
%\usepackage{kuvio,amscd,diagrams,dcpic,xymatrix,diagxy}
\usepackage{tikz,paralist,mathtools}

\newcommand{\M}{\mathfrak{M}}
\newcommand{\F}{\mathcal{F}}
\newcommand{\G}{\mathcal{G}}
\renewcommand{\P}{\mathbb{P}}
\newcommand{\E}{\mathbb{E}}
\newcommand{\R}{\mathbb{R}}
\newcommand{\B}{\mathcal{B}}
\newcommand{\N}{\mathcal{N}}
\newcommand{\1}{\mathbbm{1}}
\newcommand{\var}{\mathrm{Var}}
\newcommand{\cov}{\mathrm{Cov}}
\newcommand{\corr}{\mathrm{Corr}}
\newcommand{\pn}{\mathrm{p.n.}}
\newcommand{\sgn}{{\mathrm{sgn}}\,}
\newcommand{\bigslant}[2]{{\left.\raisebox{.2em}{$#1$}\middle/\raisebox{-.2em}{$#2$}\right.}}
\newcommand{\mf}[1]{{\mathfrak{#1}}}
\newcommand{\mb}[1]{{\mathbb{#1}}}
\newcommand{\mc}[1]{{\mathcal{#1}}}
\newcommand{\mr}[1]{{\mathrm{#1}}}
\newcommand{\ds}{\displaystyle}
\newcommand{\res}{{\mathrm{res}}}

\newcounter{punkt}

\theoremstyle{plain}
\newtheorem{theorem}[punkt]{Twierdzenie}
\newtheorem{theoremnp}[punkt]{Twierdzenie (bez dowodu)}
\newtheorem{lemma}[punkt]{Lemat}
\newtheorem{proposition}[punkt]{Stwierdzenie}

\theoremstyle{definition}
\newtheorem{definition}[punkt]{Definicja}
\newtheorem{fact}[punkt]{Fakt}
\newtheorem{corollary}[punkt]{Wniosek}

\theoremstyle{remark}
\newtheorem{remark}[punkt]{Uwaga}
\newtheorem{example}[punkt]{Przykład}
\newtheorem{exercise}[punkt]{Ćwiczenie}

\markright{Piotr Suwara\hfill Rachunek prawdopodobieństwa II: 22 stycznia 2014 \hfill}

\begin{document}
	\begin{fact}
		$\Pi$ rozkład stacjonarny wtw, gdy $\Pi = \Pi P$,
		tj. $\forall_x \pi_x = \sum_{y \in E} \pi_y p_{y,x}$.
	\end{fact}
	
	\begin{remark}
		Jeśli $\Pi$ stacjonarny, to $\forall_x \P_\Pi(X_n = x) = \pi_x$.
	\end{remark}
	
	\begin{theorem}
		Jeśli przestrzeń stanów jest skończona, to istnieje rozkład
		stacjonarny.
	\end{theorem}
	
	\begin{theorem}[ergodyczne dla skończonej przestrzeni stanów]
		$(X_n)_{n>0}$ ł.M. nieokresowy, nieprzywiedlny, 
		o~macierzy przejścia $P$ i~rozkładzie stacjonarnym $\Pi$.
		Wówczas $$\forall_{x, y \in E} 
		\lim_{n \to \infty} p_{x,y}(n) = \pi_y,$$
		a~nawet $\displaystyle \exists_{C<\infty, \gamma<1}
		|p_{x,y}(n) - \pi_y| \leq C \cdot \gamma^n$.
	\end{theorem}
	
	\begin{theorem}
		Rozkład stacjonarny w~nieprzywiedlnym ł.M. jest jednoznaczny.
	\end{theorem}
	
	\begin{proposition}
		Skończony nieprzywiedlny ł.M., $\Pi$ rozkład stacjonarny.
		Wtedy
		$\forall_x \pi_x > 0$ oraz $$\forall_x \pi_x = \frac1{\mu_x},
		\quad \mu_x = \E t_x = \E \inf \{ n \geq 1: X_n = n\}.$$
	\end{proposition}
	
	\begin{definition}[częstość przebywania w~zbiorze]
		$\displaystyle \nu_A(n) = 
			\frac1n \#\{1 \leq k \leq n : x_k \in A\}$
	\end{definition}
	
	\begin{theorem}[ergodyczne znowu]
		$(X_n)$ nieprzywiedlny nieokresowy ł.M. o~skończonej 
		przestrzeni stanów i~dowolnym rozkładzie
		początkowym, wtedy
		$\forall_{A \subset E} 
		\lim_{n \to \infty} \nu_n(A) = \sum_{x \in A} \pi_x$ p.n.,
		gdzie $\Pi$ to rozkład stacjonarny.
	\end{theorem}
	
	\begin{theorem}[ergodyczne ogólne]
		$(X_n)$ nieprzywiedlny nieodwracalny ł.M., dla którego istnieje
		rozkład stacjonarny $\Pi$. Wówczas
		\begin{enumerate}[(i)]
			\item $\Pi$ jest jedyny i~$\forall_{x, y} 
				\lim_{n \to \infty} p_{x,y}(n) = \pi_y$,
			\item ł.M. jest powracający, $\forall_x \pi_x >0$,
			$\pi_x = \frac1{\mu_x}, 
			\mu_x = \E_x \inf \{ n \geq 1: X_n = x \}$,
			\item $\forall_{A \subset E} \lim_{n \to \infty}
				\nu_n(A) = \Pi(A) = \sum_{x \in A} \pi_x$ p.n.
		\end{enumerate}

	\end{theorem}
	
	\begin{definition}[prawdopodobieństwo dojścia z $x$ do $F$]
		$\displaystyle p_F(x) = 
			\P_x \left(\exists_{n \geq 0} X_n \in F\right)$
	\end{definition}
	
	\begin{definition}[czas oczekiwania na dojście]
		$m_F(x) = \E_x \inf \{ n \geq 0: X_n \in F \}$
	\end{definition}
	
	\begin{fact}
		$p_F, m_F$ spełniają układ równań (o~jednoznacznym 
		rozwiązaniu dla $|E| < + \infty$):
		$$
		\begin{cases}
			p_F (x) = 1 & \forall_{x \in F}\\
			p_F(x) = 0 & x \nrightarrow F 
			(\iff \forall_{y\in F,n} p_{xy}(n) = 0) \\
			p_F(x) = \sum_{y \in F} p_{xy} p_F(y)
				& \forall_{x \notin F}
		\end{cases}
		$$
		$$
		\begin{cases}
			m_F (x) = 0 & \forall_{x \in F}\\
			m_F(x) = \infty & p_F(x) < 1 \\
			m_F(x) = 1 + \sum_{y \in F} p_{xy} m_F(y) 
			& \forall_{x \notin F}
		\end{cases}
		$$
	\end{fact}













\end{document}
 
 
