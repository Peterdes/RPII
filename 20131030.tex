\documentclass[12pt]{article}
\usepackage{polski}
\usepackage[utf8]{inputenc}
\usepackage[T1]{fontenc}
\usepackage{amsmath}
\usepackage{amsfonts}
\usepackage{fancyhdr}
\usepackage{lastpage}
\usepackage{multirow}
\usepackage{amssymb}
\usepackage{amsthm}
\usepackage{textcomp}
\usepackage{bbm}
\frenchspacing
\usepackage{fullpage}
\setlength{\headsep}{30pt}
\setlength{\headheight}{12pt}
%\setlength{\voffset}{-30pt}
%\setlength{\textheight}{730pt}
\pagestyle{myheadings}
%\usepackage{kuvio,amscd,diagrams,dcpic,xymatrix,diagxy}
\usepackage{tikz,paralist,mathtools}

\newcommand{\M}{\mathfrak{M}}
\newcommand{\F}{\mathcal{F}}
\newcommand{\G}{\mathcal{G}}
\renewcommand{\P}{\mathbb{P}}
\newcommand{\E}{\mathbb{E}}
\newcommand{\R}{\mathbb{R}}
\newcommand{\B}{\mathcal{B}}
\newcommand{\N}{\mathcal{N}}
\newcommand{\1}{\mathbbm{1}}
\newcommand{\var}{\mathrm{Var}}
\newcommand{\cov}{\mathrm{Cov}}
\newcommand{\corr}{\mathrm{Corr}}
\newcommand{\pn}{\mathrm{p.n.}}
\newcommand{\sgn}{{\mathrm{sgn}}\,}
\newcommand{\bigslant}[2]{{\left.\raisebox{.2em}{$#1$}\middle/\raisebox{-.2em}{$#2$}\right.}}
\newcommand{\mf}[1]{{\mathfrak{#1}}}
\newcommand{\mb}[1]{{\mathbb{#1}}}
\newcommand{\mc}[1]{{\mathcal{#1}}}
\newcommand{\mr}[1]{{\mathrm{#1}}}
\newcommand{\ds}{\displaystyle}
\newcommand{\res}{{\mathrm{res}}}

\newcounter{punkt}

\theoremstyle{plain}
\newtheorem{theorem}[punkt]{Theorem}
\newtheorem{theoremnp}[punkt]{Theorem (no proof)}
\newtheorem{lemma}[punkt]{Lemma}
\newtheorem{proposition}[punkt]{Proposition}

\theoremstyle{definition}
\newtheorem{definition}[punkt]{Definition}
\newtheorem{fact}[punkt]{Fact}
\newtheorem{corollary}[punkt]{Corollary}

\theoremstyle{remark}
\newtheorem{remark}[punkt]{Remark}
\newtheorem{example}[punkt]{Example}
\newtheorem{exercise}[punkt]{Exercise}

\markright{Piotr Suwara\hfill Rachunek prawdopodobieństwa II: 30 października 2013 \hfill}

\begin{document}
	\begin{theorem}[L\'{e}vy-Cramer simplified]
		$X_n \implies X$ iff $\varphi_{X_n} \to \varphi_X$ pointwise.
	\end{theorem}
	
	\begin{corollary}
		$X_n, X$ are random vectors in $\R^d$, then $X_n \implies X$ iff $\forall_{t \in \R^d} \langle t, X_n\rangle \implies \langle t,X\rangle$.
	\end{corollary}
	
	\begin{remark}
		We know $X \sim \N(a, \sigma^2)$, for which $\varphi_{\N(a,\sigma^2)}(t) = e^{ita - \frac{\sigma^2}{2} t^2}$ and $X \sim a + \sigma Y$ for $Y \sim \N(0,1)$.
	\end{remark}
	
	\begin{definition}[canonical Gaussian distribution]
		The canonical $d$-dimensional Gaussian distribution on $\R^d$ is the probability measure $\gamma_d$ with the density $\displaystyle \frac{d \gamma_d(x)}{dx} = \frac{1}{(2 \pi)^{\frac{d}{2}}} e^{-\frac{|x|^2}{2}}.$
		
		Equivalently, we say that a~$d$-dimensional random vector $X$ has the canonical Gaussian distribution if $X \sim \gamma_d$, i.e. $X_1, \ldots, X_d$ are independent with $\N(0,1)$-distribution. We write $X \sim \gamma_d$ or $X \sim \N(0,I_d)$.
	\end{definition}
	
	\begin{definition}[pushforward]
		$\mu$ -- a~measure on $\R^m$, $F:\R^m \to \R^d$ measurable, $F_\# \mu$ -- \emph{a~pushforward of $\mu$} is the measure on $\R^d$ given by $F_\# \mu(A) = \mu(F^{-1}(A))$.
	\end{definition}
	
	\begin{definition}[Gaussian vector]
		A~probability measure $\mu$ on $\R^d$ is called \emph{Gaussian} or \emph{normal} iff there exists affine $U:\R^m \to \R^d$ such that $\mu = U_\# \gamma_m$.
		
		A~$d$-dimensional random vector $X$ is called Gaussian if its law is Gaussian, i.e. $X \sim U Y$, $Y \sim \gamma_m$.
	\end{definition}
	
	\begin{remark}
		$X \sim UY = AY + b$, then $\E X = b, \cov(X) = A A^T$.
	\end{remark}
	
	\begin{remark}
		$X \sim AY + b$, then $\displaystyle \varphi_X(t) = e^{i \langle \E X, t \rangle - \frac{\langle \cov(x)t, t \rangle}{2}}$
	\end{remark}
	
	\begin{definition}[Gaussian again]
		We say that a~random $d$-dimensional vector $X$ (respectively, a~probability measure $\mu$ on $\R^d$) is \emph{Gaussian} iff $\displaystyle \varphi_X(t) = e^{i \langle b,t \rangle - \frac{ \langle Ct, t\rangle }{2}}$, where $b \in \R^d$, $C \in M_{d \times d}, C = C^T, C \geq 0$.
	\end{definition}
	
	\begin{proposition}
		These definitions are equivalent.
	\end{proposition}
	
	\begin{corollary}
		Every $d$-dimensional Gaussian vector is an affine image of $d$-dimensional canonical Gaussian vector.
	\end{corollary}
	
	\begin{remark}
		Equivalently, $\langle t, X\rangle$ is Gaussian for any $t \in \R^d$.
	\end{remark}
	
	\begin{corollary}
		$X_1 \sim \N(b_1, C_1), X_2 \sim \N(b_2, C_2)$ independent Gaussian, then $X_1+X_2 \sim \N(b_1 + b_2, C_1 + C_2)$ also Gaussian.
	\end{corollary}
	
	\begin{corollary}
		Affine image of a~Gaussian vector is a~Gaussian vector.
	\end{corollary}
	
	\begin{remark}
		$X \sim \N(b, C)$ has the density iff $\det(C) \neq 0$ and then $\displaystyle g(x) = \frac{1}{(2 \pi)^{\frac{d}{2}} \sqrt{\det C}} e^{-\frac{\langle C^{-1} (x-b), (x-b)\rangle}{2}}.$
	\end{remark}
	
	\begin{theorem}
		If $X$ is a~Gaussian vector in $\R^d$, then coordinates of $X$ are independent iff they are uncorrelated (i.e. $X_1, \ldots, X_d$ independent iff $\cov(X_j, X_k) = 0 $ for $j\neq k$).
	\end{theorem}
	
	\begin{remark}
		In the theorem it is important thet the whole vector is Gaussian, not only its coordinates!
	\end{remark}
	
	\begin{theorem}[CTL in the iid case]
		$X_1, X_2, \ldots$ iid random variables, $\E X_1 = a$, $\var(X_1) = \sigma^2 \in (0,\infty)$, then $\frac{X_1 + \ldots + X_n - na}{\sqrt{n}\sigma} \implies \N(0,1)$.
	\end{theorem}

















\end{document}
 
 
