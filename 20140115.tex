\documentclass[12pt]{article}
\usepackage{polski}
\usepackage[utf8]{inputenc}
\usepackage[T1]{fontenc}
\usepackage{amsmath}
\usepackage{amsfonts}
\usepackage{fancyhdr}
\usepackage{lastpage}
\usepackage{multirow}
\usepackage{amssymb}
\usepackage{amsthm}
\usepackage{textcomp}
\usepackage{bbm}
\frenchspacing
\usepackage{fullpage}
\setlength{\headsep}{30pt}
\setlength{\headheight}{12pt}
%\setlength{\voffset}{-30pt}
%\setlength{\textheight}{730pt}
\pagestyle{myheadings}
%\usepackage{kuvio,amscd,diagrams,dcpic,xymatrix,diagxy}
\usepackage{tikz,paralist,mathtools}

\newcommand{\M}{\mathfrak{M}}
\newcommand{\F}{\mathcal{F}}
\newcommand{\G}{\mathcal{G}}
\renewcommand{\P}{\mathbb{P}}
\newcommand{\E}{\mathbb{E}}
\newcommand{\R}{\mathbb{R}}
\newcommand{\B}{\mathcal{B}}
\newcommand{\N}{\mathcal{N}}
\newcommand{\1}{\mathbbm{1}}
\newcommand{\var}{\mathrm{Var}}
\newcommand{\cov}{\mathrm{Cov}}
\newcommand{\corr}{\mathrm{Corr}}
\newcommand{\pn}{\mathrm{p.n.}}
\newcommand{\sgn}{{\mathrm{sgn}}\,}
\newcommand{\bigslant}[2]{{\left.\raisebox{.2em}{$#1$}\middle/\raisebox{-.2em}{$#2$}\right.}}
\newcommand{\mf}[1]{{\mathfrak{#1}}}
\newcommand{\mb}[1]{{\mathbb{#1}}}
\newcommand{\mc}[1]{{\mathcal{#1}}}
\newcommand{\mr}[1]{{\mathrm{#1}}}
\newcommand{\ds}{\displaystyle}
\newcommand{\res}{{\mathrm{res}}}

\newcounter{punkt}

\theoremstyle{plain}
\newtheorem{theorem}[punkt]{Twierdzenie}
\newtheorem{theoremnp}[punkt]{Twierdzenie (bez dowodu)}
\newtheorem{lemma}[punkt]{Lemat}
\newtheorem{proposition}[punkt]{Stwierdzenie}

\theoremstyle{definition}
\newtheorem{definition}[punkt]{Definicja}
\newtheorem{fact}[punkt]{Fakt}
\newtheorem{corollary}[punkt]{Wniosek}

\theoremstyle{remark}
\newtheorem{remark}[punkt]{Uwaga}
\newtheorem{example}[punkt]{Przykład}
\newtheorem{exercise}[punkt]{Ćwiczenie}

\markright{Piotr Suwara\hfill Rachunek prawdopodobieństwa II: 15 stycznia 2014 \hfill}

\begin{document}
	\begin{corollary}
		Jeśli $x \leftrightarrow y$, to $x$ powracający 
		wtw, gdy $y$ powracający.
	\end{corollary}
	
	\begin{definition}[chwilowy/powracający nieprzywiedlny ł.M.]
		Jeśli ł.M. jest nieprzywiedlny, to 
		albo wszystkie stany są chwilowe (ł.M. jest \emph{chwilowy}),
		albo wszystkie stany są powracające (ł.M. jest \emph{powracający}).
	\end{definition}
	
	\begin{fact}
		$(X_n)$ powracający nieprzywiedlny ł.M., 
		wówczas $\forall_{x,y} F_{xy} = 1$.
	\end{fact}
	
	\begin{corollary}
		Nieprzywiedlny powracający ł.M. 
		o~dowolnym rozkładzie początkowym 
		odwiedza każdy stan z~prawdopodobieństwem $1$,
		tzn. 
		$\P_\Pi\left( \forall_{y \in E} \exists_n X_n = y\right) = 1$.
	\end{corollary}
	
	\begin{example}
		W~notatkach są przykłady.
	\end{example}
	
	{\bf Okresowość ł.M.}
	
	\begin{definition}[okres]
		$x \in E$, \emph{okresem} stanu $x \in E$ nazywamy liczbę $o(x) = \mr{NWD} \{ n \geq 1: p_{x,x} (n) = 0 \}$
	\end{definition}
	
	\begin{fact}
		$x \leftrightarrow y \implies o(x) = o(y)$
	\end{fact}
	
	\begin{corollary}
		Jeśli ł.M. jest nieprzywiedlny, to wszystkie stany mają ten 
		sam okres.
	\end{corollary}
	
	\begin{definition}[okres]
		$(X_n)$ nieprzywiedly ł.M. \emph{Okresem} takiego łańcucha
		nazywamy okres dowolnego jego stanu.
		
		Mówimy, że łańcuch jest \emph{okresowy}, jeśli ma
		okres większy niż $1$, a~\emph{nieokresowy}, jeśli ma
		okres równy $1$.
	\end{definition}
	
	\begin{fact}
		Jeśli nieprzywiedlny ł.M. jest nieokresowy, to
		$\forall_{x,y} \exists_{n_0} \forall_{n \geq n_0} 
			p_{x,y}(n) > 0$.
	\end{fact}
	
	\begin{corollary}
		Dla nieprzywiedlnego nieokresowego ł.M. o~skończonej 
		przestrzeni stanów $E$ istnieje $n_0$ takie, że
		$\forall_{n \geq n_0} \forall_{x,y} p_{x,y}(n) > 0$.
	\end{corollary}
	
	{\bf Rozkłady stacjonarne}
	\begin{definition}
		Rozkład probabilistyczny $\Pi = (\pi_x)_{x \in E}$
		nazywamy stacjonarnym dla ł.M. o~macierzy przejścia
		$P = (p_{x,y})_{x,y \in E}$, jeśli 
		$\forall_x \P_\Pi(X_1 = x) = \pi_x$.
	\end{definition}














\end{document}
 
 
