\input{includepl.tex}

\markright{Piotr Suwara\hfill Rachunek prawdopodobieństwa II: 15 stycznia 2014 \hfill}

\begin{document}
	\begin{corollary}
		Jeśli $x \leftrightarrow y$, to $x$ powracający 
		wtw, gdy $y$ powracający.
	\end{corollary}
	
	\begin{definition}[chwilowy/powracający nieprzywiedlny ł.M.]
		Jeśli ł.M. jest nieprzywiedlny, to 
		albo wszystkie stany są chwilowe (ł.M. jest \emph{chwilowy}),
		albo wszystkie stany są powracające (ł.M. jest \emph{powracający}).
	\end{definition}
	
	\begin{fact}
		$(X_n)$ powracający nieprzywiedlny ł.M., 
		wówczas $\forall_{x,y} F_{xy} = 1$.
	\end{fact}
	
	\begin{corollary}
		Nieprzywiedlny powracający ł.M. 
		o~dowolnym rozkładzie początkowym 
		odwiedza każdy stan z~prawdopodobieństwem $1$,
		tzn. 
		$\P_\Pi\left( \forall_{y \in E} \exists_n X_n = y\right) = 1$.
	\end{corollary}
	
	\begin{example}
		W~notatkach są przykłady.
	\end{example}
	
	{\bf Okresowość ł.M.}
	
	\begin{definition}[okres]
		$x \in E$, \emph{okresem} stanu $x \in E$ nazywamy liczbę \\ 
		$o(x) = \mr{NWD} \{ n \geq 1: p_{x,x} (n) = 0 \}$
	\end{definition}
	
	\begin{fact}
		$x \leftrightarrow y \implies o(x) = o(y)$
	\end{fact}
	
	\begin{corollary}
		Jeśli ł.M. jest nieprzywiedlny, to wszystkie stany mają ten 
		sam okres.
	\end{corollary}
	
	\begin{definition}[okres]
		$(X_n)$ nieprzywiedly ł.M. \emph{Okresem} takiego łańcucha
		nazywamy okres dowolnego jego stanu.
		
		Mówimy, że łańcuch jest \emph{okresowy}, jeśli ma
		okres większy niż $1$, a~\emph{nieokresowy}, jeśli ma
		okres równy $1$.
	\end{definition}
	
	\begin{fact}
		Jeśli nieprzywiedlny ł.M. jest nieokresowy, to
		$\forall_{x,y} \exists_{n_0} \forall_{n \geq n_0}\, 
			p_{x,y}(n) > 0$.
	\end{fact}
	
	\begin{corollary}
		Dla nieprzywiedlnego nieokresowego ł.M. o~skończonej 
		przestrzeni stanów $E$ istnieje $n_0$ takie, że
		$\forall_{n \geq n_0} \forall_{x,y}\, p_{x,y}(n) > 0$.
	\end{corollary}
	
	{\bf Rozkłady stacjonarne}
	\begin{definition}
		Rozkład probabilistyczny $\Pi = (\pi_x)_{x \in E}$
		nazywamy stacjonarnym dla ł.M. o~macierzy przejścia
		$P = (p_{x,y})_{x,y \in E}$, jeśli 
		$\forall_x\, \P_\Pi(X_1 = x) = \pi_x$.
	\end{definition}














\end{document}
 
 
