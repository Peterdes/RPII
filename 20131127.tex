\documentclass[12pt]{article}
\usepackage{polski}
\usepackage[utf8]{inputenc}
\usepackage[T1]{fontenc}
\usepackage{amsmath}
\usepackage{amsfonts}
\usepackage{fancyhdr}
\usepackage{lastpage}
\usepackage{multirow}
\usepackage{amssymb}
\usepackage{amsthm}
\usepackage{textcomp}
\usepackage{bbm}
\frenchspacing
\usepackage{fullpage}
\setlength{\headsep}{30pt}
\setlength{\headheight}{12pt}
%\setlength{\voffset}{-30pt}
%\setlength{\textheight}{730pt}
\pagestyle{myheadings}
%\usepackage{kuvio,amscd,diagrams,dcpic,xymatrix,diagxy}
\usepackage{tikz,paralist,mathtools}

\newcommand{\M}{\mathfrak{M}}
\newcommand{\F}{\mathcal{F}}
\newcommand{\G}{\mathcal{G}}
\renewcommand{\P}{\mathbb{P}}
\newcommand{\E}{\mathbb{E}}
\newcommand{\R}{\mathbb{R}}
\newcommand{\B}{\mathcal{B}}
\newcommand{\N}{\mathcal{N}}
\newcommand{\1}{\mathbbm{1}}
\newcommand{\var}{\mathrm{Var}}
\newcommand{\cov}{\mathrm{Cov}}
\newcommand{\corr}{\mathrm{Corr}}
\newcommand{\pn}{\mathrm{p.n.}}
\newcommand{\sgn}{{\mathrm{sgn}}\,}
\newcommand{\bigslant}[2]{{\left.\raisebox{.2em}{$#1$}\middle/\raisebox{-.2em}{$#2$}\right.}}
\newcommand{\mf}[1]{{\mathfrak{#1}}}
\newcommand{\mb}[1]{{\mathbb{#1}}}
\newcommand{\mc}[1]{{\mathcal{#1}}}
\newcommand{\mr}[1]{{\mathrm{#1}}}
\newcommand{\ds}{\displaystyle}
\newcommand{\res}{{\mathrm{res}}}

\newcounter{punkt}

\theoremstyle{plain}
\newtheorem{theorem}[punkt]{Theorem}
\newtheorem{theoremnp}[punkt]{Theorem (no proof)}
\newtheorem{lemma}[punkt]{Lemma}
\newtheorem{proposition}[punkt]{Proposition}

\theoremstyle{definition}
\newtheorem{definition}[punkt]{Definition}
\newtheorem{fact}[punkt]{Fact}
\newtheorem{corollary}[punkt]{Corollary}

\theoremstyle{remark}
\newtheorem{remark}[punkt]{Remark}
\newtheorem{example}[punkt]{Example}
\newtheorem{exercise}[punkt]{Exercise}

\markright{Piotr Suwara\hfill Rachunek prawdopodobieństwa II: 27 listopada 2013 \hfill}

\begin{document}
	\begin{definition}[filtration]
		$(\F_t)_T$, for $T$ a~segment in $\mb{Z}$, is a~\emph{filtration},
		if $\F_t \subset \F$ is a~$\sigma$-body and $\forall_{t \leq s} \F_t \subset \F_s$.
	\end{definition}
	
	\begin{definition}[stopping time]
		$\tau: \omega \to T \cup \{\infty\}$ is a~\emph{stopping time}, if
		$\forall_{t \in T} \{\tau \leq t\} \in \F_t$
		($\iff \forall_{t \in T} \{ \tau = t\} \in \F_t\}$).
	\end{definition}
	
	\begin{definition}
		Let $(\F_t)$ be a filtration, $\tau$ be a~stopping time, then define
		$$\F_\tau = \{ A \in \F: A \cap \{ \tau \leq t\} \in \F_t\}.$$
	\end{definition}
	
	\begin{proposition}
		$\F_\tau = \{ A \in \F: A \cap \{ \tau = t \} \in \F_t\}$.
	\end{proposition}
	
	\begin{proposition}
		$\tau_1, \tau_2$ stopping times, then $\tau_1 \wedge \tau2 = \min(\tau_1, \tau_2)$
		and $\tau_1 \vee \tau_2 = \max(\tau_1, \tau_2)$ are too.
		
		$\tau = t$ is a~stopping time.
		
		$\tau_1 \leq \tau_2$ stopping times $\implies \F_{\tau_1} \subset \F_{\tau_2}$
		
		$\tau$ jest $\F_\tau$-mierzalne.
	\end{proposition}
	
	\begin{definition}[adapted process]
		$(X_t)_{t \in T}$ is \emph{adapted to the filtration} $(\F_t)_{t \in T}$ or just \emph{$(\F_t)$-adapted}, if $\forall_t$ $X_t$ is $\F_t$-measurable.
	\end{definition}
	
	\begin{proposition}
		$(\F_t)$ filtration, $(X_t)$ is $(\F_t)$-adapted, $\tau$ a~stopping time,
		then $\tau < \infty \implies X_\tau$ is $\F_\tau$-measurable.
		
		More generally, $X_\tau$ is $\F_\tau$-measurable on the set $\{\tau < \infty\}$,
		i.e. $\forall_{B \in \B(\R)} \{X_\tau \in B\} \cap \{t < \infty\} \in \F_\tau$.
	\end{proposition}
	
	\begin{definition}[martingale]
		$(X_t)$ is a~\emph{martingale} (resp. submartingale, supermartingale)
		with respect to a~foltration $(\F_t)$, if 
		\begin{itemize}
			\item $\forall_{t \in T}$ $X_t$ is $\F_t$-measurable,
			\item $\forall_{t \in T} \E |X_t| < \infty$,
			\item $\forall_{s \leq t, s, t \in T} \E(X_t | \F_s) = X_s$ a.s. (resp. $\geq$, $\leq$).
		\end{itemize}
	\end{definition}
	
	\begin{remark}
		$X_t$ is a~martingale iff it is both a~submartingale and a~supermartingale.
	\end{remark}
	
	\begin{remark}
		$(X_t)$ is a~$(\F_t)$-martingale if $X_t$ is $\F_t$-measurable, integrable and $\forall_{s < t, A \in \F_s} \E(X_s \1_A) = \E(X_t \1_A)$ (resp. $\leq$ for submartingale, $\geq$ for supermartingale).
	\end{remark}
	
	\begin{remark}
		For $T$ a~segment in $\mb{Z}$, $(X_t)$ is $(\F_t)$-martingale iff
		$X_t$ is $\F_t$-measurable, $\E|X_t|<\infty$, $\E(X_{s+1} | \F_s) = X_s$ a.s. (resp. $\geq$ for submartingale, $\leq$ for supermartingale).
	\end{remark}
	
	\begin{remark}
		$X_t$ submartingale iff $-X_t$ supermartingale.
	\end{remark}
	
	\begin{remark}
		$X_t, Y_t$ are $\F_t$-martingales, then $a X_t + b Y_t$ also (for submartingale take $a, b \geq 0$).
	\end{remark}
	
	\begin{definition}
		$(F_n)_{n \geq 0}$ filtration generated by $(X_1, X_2, \ldots)$, $\F_0 = \{ \varnothing, \Omega\}, \F_n = \sigma(X_1, \ldots, X_n)$.
	\end{definition}











\end{document}
 
 
