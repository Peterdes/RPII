\documentclass[12pt]{article}
\usepackage{polski}
\usepackage[utf8]{inputenc}
\usepackage[T1]{fontenc}
\usepackage{amsmath}
\usepackage{amsfonts}
\usepackage{fancyhdr}
\usepackage{lastpage}
\usepackage{multirow}
\usepackage{amssymb}
\usepackage{amsthm}
\usepackage{textcomp}
\usepackage{bbm}
\frenchspacing
\usepackage{fullpage}
\setlength{\headsep}{30pt}
\setlength{\headheight}{12pt}
%\setlength{\voffset}{-30pt}
%\setlength{\textheight}{730pt}
\pagestyle{myheadings}
%\usepackage{kuvio,amscd,diagrams,dcpic,xymatrix,diagxy}
\usepackage{tikz,paralist,mathtools}

\newcommand{\M}{\mathfrak{M}}
\newcommand{\F}{\mathcal{F}}
\newcommand{\G}{\mathcal{G}}
\renewcommand{\P}{\mathbb{P}}
\newcommand{\E}{\mathbb{E}}
\newcommand{\R}{\mathbb{R}}
\newcommand{\B}{\mathcal{B}}
\newcommand{\N}{\mathcal{N}}
\newcommand{\1}{\mathbbm{1}}
\newcommand{\var}{\mathrm{Var}}
\newcommand{\cov}{\mathrm{Cov}}
\newcommand{\corr}{\mathrm{Corr}}
\newcommand{\pn}{\mathrm{p.n.}}
\newcommand{\sgn}{{\mathrm{sgn}}\,}
\newcommand{\bigslant}[2]{{\left.\raisebox{.2em}{$#1$}\middle/\raisebox{-.2em}{$#2$}\right.}}
\newcommand{\mf}[1]{{\mathfrak{#1}}}
\newcommand{\mb}[1]{{\mathbb{#1}}}
\newcommand{\mc}[1]{{\mathcal{#1}}}
\newcommand{\mr}[1]{{\mathrm{#1}}}
\newcommand{\ds}{\displaystyle}
\newcommand{\res}{{\mathrm{res}}}

\newcounter{punkt}

\theoremstyle{plain}
\newtheorem{theorem}[punkt]{Twierdzenie}
\newtheorem{theoremnp}[punkt]{Twierdzenie (bez dowodu)}
\newtheorem{lemma}[punkt]{Lemat}
\newtheorem{proposition}[punkt]{Stwierdzenie}

\theoremstyle{definition}
\newtheorem{definition}[punkt]{Definicja}
\newtheorem{fact}[punkt]{Fakt}
\newtheorem{corollary}[punkt]{Wniosek}

\theoremstyle{remark}
\newtheorem{remark}[punkt]{Uwaga}
\newtheorem{example}[punkt]{Przykład}
\newtheorem{exercise}[punkt]{Ćwiczenie}

\markright{Piotr Suwara\hfill Rachunek prawdopodobieństwa II: 8 stycznia 2014 \hfill}

\begin{document}
	\begin{definition}
		$\P_x(A) = \P(A | X_0 = x)$ o~ile $\P(X_0 = x) > 0$.
	\end{definition}
	
	\begin{fact}
		$(X_n)$ jednorodny ł.M. na $(\Omega, \F, \P)$ 
		z~macierzą przejścia $P$ taką, że $\P(X_0 = x)>0$.
		Wtedy względem $\P_x$, $(X_n)_{n \geq 0}$ jest jednorodnym
		ł. M. z~macierzą przejścia $P$, o~rozkładzie poczatkowym 
		$\delta_x$, tzn. $\P_x(X_0 = x) = \delta_x(x) = 1$.
	\end{fact}
	
	\begin{definition}
		$\Pi$ układ probabilistyczny na $E$, wtedy 
		$P_\Pi(A) = \sum_x \pi_x \P_x$.
	\end{definition}
	
	\begin{fact}
		$(X_n)$ j.ł.M. w~$(\Omega, \F, \P)$ taki, że 
		$\forall_x \P(X_n = x) > 0$.
		Wtedy dla każdego rozkładu $\Pi$ na $E$ ciąg $(X_n)$ jest 
		j.ł.M. względem $\P_\Pi$ z~macierzą przejścia $P$
		i~rozkładem początkowym $\Pi$.
	\end{fact}
	
	\begin{fact}
		$(X_n)_{n \geq 0}$ j.ł.M. z~macierzą przejścia $P$, wtedy
		\begin{itemize}
			\item $\P_x(X_1 = x_1, \ldots, X_n = x_n)
			= p_{x x_1} p_{x_1 x_2} \cdot \ldots\cdot p_{x_{n-1}x_n}$,
			\item $\P_x (X_1 = x_1, \ldots, X_{n+m} = x_{n+m})$
			\\${= \P_x (X_1 = x_1, \ldots, X_n = x_n)
			\cdot \P_{x_n}(X_1 = x_{n+1}, \ldots, X_m = x_{n+m})}$,
			\item $\forall_{I \subset E^{n-1}, J \subset E^m} \:
			\P_x( (X_1, \ldots, X_{n-1}) \in I, 
				X_n = x_n, 
				(X_{n+1}, \ldots, X_{n+m}) \in J )$\\
			${= \P_x( (X_1, \ldots, X_{n-1}) \in I, X_n = x_n)
				\cdot \P_{x_n}( (X_1, \ldots, X_m) \in J)}$.
		\end{itemize}

	\end{fact}
	
	\begin{definition}[macierz przejścia w~$n$ krokach]
		$P(n) = (p_{x,y}(n))$, 
		\\ gdzie $p_{x,y}(n) = \P(X_n = y | X_0 = x) = \P_x(X_n = y)$.
	\end{definition}
	
	\begin{remark}
		$p_{x,y}(n+m) = \sum_z p_{x,z}(n) p_{z,y}(m)$.
	\end{remark}
	
	\begin{remark}
		$P(0) = P^0 = \mr{Id}$
	\end{remark}
	
	\begin{remark}
		$f_{x,y}(n) = \P_x(X_1 \neq y, \ldots, X_{n-1} \neq y, X_n = y)$, \\
		wtedy $p_{x,y}(n) = \sum_{m=1}^n f_{x,y}(m) p_{y,y}(n-m)$.
	\end{remark}
	
	\pagebreak
	
	{\bf Klasyfikacja stanów}
	\begin{definition}
		\begin{enumerate}[A.]
			\item \emph{Ze stanu $x$ da się dojść do stanu $y$},
			ozn. $x \to y$, jeśli $\exists_{n \geq 0} p_{x,y}(n) > 0$.
			\item Stany $x, y$ się \emph{komunikują},
			jesli $x \to y$ oraz $y \to x$.
			\item Ł.M. jest \emph{nieprzywiedlny}, 
			jeśli każde dwa stany się komunikują.
			\item Stan $x$ jest \emph{nieistotny}, jeśli
			$\exists_y\, x \to y \wedge y \nrightarrow x$.
			\item Stan $x$ jest \emph{pochłaniający}, jeśli
			$p_{x,x} = 1$.
			\item Zbiór stanów $C \subset E$ jest \emph{zamknięty},
			jeśli $\forall_{x \in C}\, x \to y \implies y \in C$.
		\end{enumerate}

	\end{definition}
	
	{\bf Stany chwilowe i~powracające}
	\begin{definition}
		$\displaystyle F_{x,y} = \sum_{n=1}^\infty f_{x,y}(n) 
		= \sum \P_x\left( \exists_{n \geq 1} X_n=y \right)$.
	\end{definition}
	
	\begin{definition}
		Mówimy, że stan $x$ jest:
		\begin{enumerate}[a)]
			\item \emph{chwilowy}, jeśli $F_{xx} < 1$,
			\item \emph{powracający}, jeśli $F_{xx} = 1$.
		\end{enumerate}

	\end{definition}
	
	\begin{remark}
		$x \to y, y \to z \implies x \to z$.
	\end{remark}
	
	\begin{remark}
		$\leftrightarrow$ to relacja równoważności.
	\end{remark}
	
	\begin{remark}
		$C$ zamknięty, to można rozpatrywać ł.M. o~zbiorze stanów $C$.
	\end{remark}
	
	\begin{remark}
		Ł.M. jest nieprzywiedlny wtw, gdy jedyne zamknięte zbiory 
		to $\varnothing, E$.
	\end{remark}
	
	\begin{definition}[liczba wizyt w~stanie $x$]
		$N_x = \sum_{n=1}^\infty \1_{\{ X_n = X \}}$
	\end{definition}
	
	\begin{fact}
		Dla $k \geq 1$, $\P_x(N_y \geq k) = F_{xy} F_{yy}^{k-1}$.
	\end{fact}
	
	\begin{corollary}
		Jeśli $x$ chwilowy, to $\P_x(N_x = \infty) = 0$, 
		czyli $\P_x(N_x < \infty) = 1$.
	\end{corollary}
	
	\begin{corollary}
		Jeśli $x$ powracający, to $\P_x(N_x = \infty) = 1$.
	\end{corollary}
	
	\begin{theorem}[kryterium powracalności]
		Stan $x$ jest \emph{powracający} 
		wtw, gdy ${\sum_n p_{xx}(n) = \infty}$.
		
		Stan $x$ jest \emph{chwilowy}
		wtw, gdy $\sum_n p_{xx}(n) < \infty$.
	\end{theorem}
	
	\begin{fact}
		$\displaystyle \sum_n p_{xx}(n) = \frac1{1-F_{xx}}$
	\end{fact}





















\end{document}
 
 
