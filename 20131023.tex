\documentclass[12pt]{article}
\usepackage{polski}
\usepackage[utf8]{inputenc}
\usepackage[T1]{fontenc}
\usepackage{amsmath}
\usepackage{amsfonts}
\usepackage{fancyhdr}
\usepackage{lastpage}
\usepackage{multirow}
\usepackage{amssymb}
\usepackage{amsthm}
\usepackage{textcomp}
\usepackage{bbm}
\frenchspacing
\usepackage{fullpage}
\setlength{\headsep}{30pt}
\setlength{\headheight}{12pt}
%\setlength{\voffset}{-30pt}
%\setlength{\textheight}{730pt}
\pagestyle{myheadings}
%\usepackage{kuvio,amscd,diagrams,dcpic,xymatrix,diagxy}
\usepackage{tikz,paralist,mathtools}

\newcommand{\M}{\mathfrak{M}}
\newcommand{\F}{\mathcal{F}}
\newcommand{\G}{\mathcal{G}}
\renewcommand{\P}{\mathbb{P}}
\newcommand{\E}{\mathbb{E}}
\newcommand{\R}{\mathbb{R}}
\newcommand{\B}{\mathcal{B}}
\newcommand{\N}{\mathcal{N}}
\newcommand{\1}{\mathbbm{1}}
\newcommand{\var}{\mathrm{Var}}
\newcommand{\cov}{\mathrm{Cov}}
\newcommand{\corr}{\mathrm{Corr}}
\newcommand{\pn}{\mathrm{p.n.}}
\newcommand{\sgn}{{\mathrm{sgn}}\,}
\newcommand{\bigslant}[2]{{\left.\raisebox{.2em}{$#1$}\middle/\raisebox{-.2em}{$#2$}\right.}}
\newcommand{\mf}[1]{{\mathfrak{#1}}}
\newcommand{\mb}[1]{{\mathbb{#1}}}
\newcommand{\mc}[1]{{\mathcal{#1}}}
\newcommand{\mr}[1]{{\mathrm{#1}}}
\newcommand{\ds}{\displaystyle}
\newcommand{\res}{{\mathrm{res}}}

\newcounter{punkt}

\theoremstyle{plain}
\newtheorem{theorem}[punkt]{Theorem}
\newtheorem{theoremnp}[punkt]{Theorem (no proof)}
\newtheorem{lemma}[punkt]{Lemma}
\newtheorem{proposition}[punkt]{Proposition}

\theoremstyle{definition}
\newtheorem{definition}[punkt]{Definition}
\newtheorem{fact}[punkt]{Fact}
\newtheorem{corollary}[punkt]{Corollary}

\theoremstyle{remark}
\newtheorem{remark}[punkt]{Remark}
\newtheorem{example}[punkt]{Example}
\newtheorem{exercise}[punkt]{Exercise}

\markright{Piotr Suwara\hfill Rachunek prawdopodobieństwa II: 23 października 2013 \hfill}

\begin{document}
	\begin{theorem}
		$\displaystyle \varphi_{\N(0,1)}(t) = e^{-\frac{t^2}{2}} \qquad \varphi_{\N(a,\sigma^2)}(t) = e^{ita - \frac{t^2 \sigma^2}{2}}$
	\end{theorem}
	
	\begin{theorem}
		$X$ random variable in $\R^d$, $\E |X_1|^{k_1} \ldots |X_d|^{k_d} < \infty$, then $\displaystyle \frac{\partial^{k_1}}{\partial t_1^{k_1}} \ldots \frac{\partial^{k_d}}{\partial t_d^{k_d}} \varphi_X(t)$ exists and equals $\displaystyle i^{|k|} \E X_1^{k_1} \ldots X_d^{k_d}$.
	\end{theorem}
	
	\begin{remark}
		$X_1, \ldots, X_d$ independent, then $\varphi_{X_1 + \ldots + X_d}(t) = \varphi_{X_1}(t) \ldots \varphi_{X_d}(t)$, but opposite is not true in general.
	\end{remark}
	
	\begin{theorem}
		$X_1, \ldots, X_d$ independent iff $\forall_{t \in \R^d} \varphi_{(X_1, \ldots, X_d)}(t) = \varphi_{X_1}(t_1) \ldots \varphi_{X_d}(t_d)$.
	\end{theorem}
	
	\begin{theorem}[L\'{e}vy-Cramer]
		1. If $\mu_n, \mu$ probability measures on $\R^d$ and $\mu_n \implies \mu$, then $\displaystyle\forall_{t \in \R^d} \varphi_{\mu_n}(t) \to \varphi_\mu(t)$.
		
		2. If $\mu_n$ probability measure on $\R^d$ and there exists function $\varphi:\R^d \to \mb{C}$ such that $\forall_{t \in \R^d} \varphi_{\mu_n}(t) \to \varphi(t)$ and $\varphi$ is continuous at 0, then there exists a~probability measure $\mu$ on $\R^d$ such that $\varphi=\varphi_\mu$ and $\mu_n \implies \mu$.
	\end{theorem}
	
	\begin{corollary}
		$\mu_n \implies \mu$ iff $\varphi_{\mu_n} \to \varphi_\mu$ pointwise.
	\end{corollary}
	
	\begin{theorem}[Inverse Fourier Theorem]
		Suppose that $\mu$ is a~probability measure on $\R^d$ and $\varphi_\mu \in L^1(\R^d)$ such that $\int_{\R^d} |\varphi_\mu(x)|\,dx < \infty$, then $\mu$ has the density $g$ given by the formula $$g(x) = \frac{1}{(2 \pi)^d} \int_{\R^d} \varphi_\mu(t) e^{-i\langle t,x\rangle} dt.$$
	\end{theorem}






\end{document}
 
 
