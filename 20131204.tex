\documentclass[12pt]{article}
\usepackage{polski}
\usepackage[utf8]{inputenc}
\usepackage[T1]{fontenc}
\usepackage{amsmath}
\usepackage{amsfonts}
\usepackage{fancyhdr}
\usepackage{lastpage}
\usepackage{multirow}
\usepackage{amssymb}
\usepackage{amsthm}
\usepackage{textcomp}
\usepackage{bbm}
\frenchspacing
\usepackage{fullpage}
\setlength{\headsep}{30pt}
\setlength{\headheight}{12pt}
%\setlength{\voffset}{-30pt}
%\setlength{\textheight}{730pt}
\pagestyle{myheadings}
%\usepackage{kuvio,amscd,diagrams,dcpic,xymatrix,diagxy}
\usepackage{tikz,paralist,mathtools}

\newcommand{\M}{\mathfrak{M}}
\newcommand{\F}{\mathcal{F}}
\newcommand{\G}{\mathcal{G}}
\renewcommand{\P}{\mathbb{P}}
\newcommand{\E}{\mathbb{E}}
\newcommand{\R}{\mathbb{R}}
\newcommand{\B}{\mathcal{B}}
\newcommand{\N}{\mathcal{N}}
\newcommand{\1}{\mathbbm{1}}
\newcommand{\var}{\mathrm{Var}}
\newcommand{\cov}{\mathrm{Cov}}
\newcommand{\corr}{\mathrm{Corr}}
\newcommand{\pn}{\mathrm{p.n.}}
\newcommand{\sgn}{{\mathrm{sgn}}\,}
\newcommand{\bigslant}[2]{{\left.\raisebox{.2em}{$#1$}\middle/\raisebox{-.2em}{$#2$}\right.}}
\newcommand{\mf}[1]{{\mathfrak{#1}}}
\newcommand{\mb}[1]{{\mathbb{#1}}}
\newcommand{\mc}[1]{{\mathcal{#1}}}
\newcommand{\mr}[1]{{\mathrm{#1}}}
\newcommand{\ds}{\displaystyle}
\newcommand{\res}{{\mathrm{res}}}

\newcounter{punkt}

\theoremstyle{plain}
\newtheorem{theorem}[punkt]{Twierdzenie}
\newtheorem{theoremnp}[punkt]{Twierdzenie (bez dowodu)}
\newtheorem{lemma}[punkt]{Lemat}
\newtheorem{proposition}[punkt]{Stwierdzenie}

\theoremstyle{definition}
\newtheorem{definition}[punkt]{Definicja}
\newtheorem{fact}[punkt]{Fakt}
\newtheorem{corollary}[punkt]{Wniosek}

\theoremstyle{remark}
\newtheorem{remark}[punkt]{Uwaga}
\newtheorem{example}[punkt]{Przykład}
\newtheorem{exercise}[punkt]{Ćwiczenie}

\markright{Piotr Suwara\hfill Rachunek prawdopodobieństwa II: 4 grudnia 2013 \hfill}

\begin{document}
	\begin{proposition}
		$X_k$ jest $\F_k$-adaptowalny, $\tau$ moment zatrzymania, 
		wtedy $X_\tau$ jest $\F_\tau$-mierzalne na $\{ \tau < \infty\}$.
	\end{proposition}
	
	\begin{theorem}[Doob optional sampling]
		$(X_n, \F_n)_{n \geq 0}$ (nad, pod)martyngał,
		$\tau_1 \leq \tau_2 \leq N < \infty$ dwa momenty zatrzymania.
		Wtedy $(X_{\tau_i}, \F_{\tau_i})$ jest (nad, pod)martyngałem,
		tzn. $\E(X_{\tau_2}|\F_{\tau_1}) (\leq, \geq)= X_{\tau_1}$ p.n.
		
		W~szczególności $\E X_{\tau _2} (\leq,\geq)= \E X_{\tau_1}$.
	\end{theorem}
	
	\begin{remark}
		Założenie $\tau_2 \leq N < \infty$ jest kluczowe!
	\end{remark}
	
	\begin{theorem}[tożsamość Walda]
		$X_1, X_2, \ldots$ iid, $\E |X_1| < \infty$, 
		$S_0 = 0, S_n = X_1 + \ldots + X_n$, 
		$\F_0=\{\varnothing, \Omega\}, \F_n=\sigma(X_1, \ldots, X_n)$,
		$\tau$ moment zatrzymania względem $\F_n$ 
		taki, że $\E \tau < \infty$.
		
		Wtedy $\E S_\tau = \E \tau \E X_1$.
	\end{theorem}
	
	\begin{corollary}
		$S_n$ jak wyżej, $a \in \mb{Z}$, $\tau_a  = \inf\{n : S_n = a\}$.
		Wówczas $\tau_a < \infty$ p.n., czyli symetryczne błądzenie losowe na $\mb{Z}$ z~prawdopodobieństwem~$1$ odwiedza każdy punkt $\mb{Z}$.
	\end{corollary}
	
	\begin{theorem}[o zbieżności p.n. dla martyngałów]
		$(X_n, \F_n)_{n \geq 0}$ nadmartyngał 
		taki, że $\sup_n \E X_n^- < \infty$. 
		Wówczas $X = \lim_{n \to \infty} X_n$ istnieje p.n. 
		oraz $\E |X| < \infty$.
	\end{theorem}
	
	Poniższe służą dowodowi twierdzenia.
	
	\begin{fact}
		$(x_n)$ ciąg, wtedy $\lim x_n$ istnieje w~szerszym sensie 
		(tzn. być może jest nieskończona) wtw, gdy 
		$\forall_{a < b, a, b \in \mb{Q}} U_a^b((x_n)) < \infty$,
		gdzie $U_a^b$ to liczba przejść w~górę przez przedział 
		$[a,b]$ dla ciągu $(x_n)$.
	\end{fact}
	
	\begin{lemma}
		$(X_n)_{n=0}^m$ nadmartyngał, $U_a^b(m)$ liczba przejść 
		przez $[a,b]$ dla $(X_n)$ do chwili $m$.
		Wtedy 
		$\E U_a^b(m) 
		\leq 
		\frac{1}{b-a} \E (X_m - a)^-
		\leq
		\frac{1}{b-a} (\E X_m^- + a^+)$
	\end{lemma}
	
	\begin{fact}
		Dla nadmartyngału $(X_n)_{n \geq 0}$ NWSR:
		\begin{itemize}
			\item $\sup_n \E |X_n| < \infty$,
			\item $\sup_n \E X_n^- < \infty$,
			\item $\lim_n \E X_n^- < \infty$.
		\end{itemize}

	\end{fact}









\end{document}
 
 
