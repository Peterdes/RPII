\documentclass[12pt]{article}
\usepackage{polski}
\usepackage[utf8]{inputenc}
\usepackage[T1]{fontenc}
\usepackage{amsmath}
\usepackage{amsfonts}
\usepackage{fancyhdr}
\usepackage{lastpage}
\usepackage{multirow}
\usepackage{amssymb}
\usepackage{amsthm}
\usepackage{textcomp}
\usepackage{bbm}
\frenchspacing
\usepackage{fullpage}
\setlength{\headsep}{30pt}
\setlength{\headheight}{12pt}
%\setlength{\voffset}{-30pt}
%\setlength{\textheight}{730pt}
\pagestyle{myheadings}
%\usepackage{kuvio,amscd,diagrams,dcpic,xymatrix,diagxy}
\usepackage{tikz,paralist,mathtools}

\newcommand{\M}{\mathfrak{M}}
\newcommand{\F}{\mathcal{F}}
\newcommand{\G}{\mathcal{G}}
\renewcommand{\P}{\mathbb{P}}
\newcommand{\E}{\mathbb{E}}
\newcommand{\R}{\mathbb{R}}
\newcommand{\B}{\mathcal{B}}
\newcommand{\N}{\mathcal{N}}
\newcommand{\1}{\mathbbm{1}}
\newcommand{\var}{\mathrm{Var}}
\newcommand{\cov}{\mathrm{Cov}}
\newcommand{\corr}{\mathrm{Corr}}
\newcommand{\pn}{\mathrm{p.n.}}
\newcommand{\sgn}{{\mathrm{sgn}}\,}
\newcommand{\bigslant}[2]{{\left.\raisebox{.2em}{$#1$}\middle/\raisebox{-.2em}{$#2$}\right.}}
\newcommand{\mf}[1]{{\mathfrak{#1}}}
\newcommand{\mb}[1]{{\mathbb{#1}}}
\newcommand{\mc}[1]{{\mathcal{#1}}}
\newcommand{\mr}[1]{{\mathrm{#1}}}
\newcommand{\ds}{\displaystyle}
\newcommand{\res}{{\mathrm{res}}}

\newcounter{punkt}

\theoremstyle{plain}
\newtheorem{theorem}[punkt]{Theorem}
\newtheorem{theoremnp}[punkt]{Theorem (no proof)}
\newtheorem{lemma}[punkt]{Lemma}
\newtheorem{proposition}[punkt]{Proposition}

\theoremstyle{definition}
\newtheorem{definition}[punkt]{Definition}
\newtheorem{fact}[punkt]{Fact}
\newtheorem{corollary}[punkt]{Corollary}

\theoremstyle{remark}
\newtheorem{remark}[punkt]{Remark}
\newtheorem{example}[punkt]{Example}
\newtheorem{exercise}[punkt]{Exercise}

\markright{Piotr Suwara\hfill Rachunek prawdopodobieństwa II: 6 listopada 2013 \hfill}

\begin{document}
	\begin{theorem}[CTL in the iid case]
		$X_1, X_2, \ldots$ iid random variables, $\E X_1 = a$, $\var(X_1) = \sigma^2 \in (0,\infty)$, then $\frac{X_1 + \ldots + X_n - na}{\sqrt{n}\sigma} \implies \N(0,1)$.
	\end{theorem}
	
	\begin{corollary}[de Moivre-Laplace]
		$X_n \sim \mr{Bin}(n,p)$, then $\frac{X_n - np}{\sqrt{np(1-p)}} \implies \N(0,1)$.
	\end{corollary}
	
	\begin{definition}[triangular array]
		$(X_{n,k})_{n=1,2,\ldots; 1 \leq k \leq k_n}$ is a~\emph{triangular array of random variables} if $\forall_n X_{n,1}, \ldots, X_{n,k_n}$ are independent.
	\end{definition}
	
	\begin{theorem}[CLT -- Lindeberg's version]
		$(X_{n,k})$ a~triangular array, $S_n = X_{n,1} + \ldots + X_{n,k_n}$,
		\begin{enumerate}
			\item $\E S_n \to m$
			\item $\var(S_n) \to \sigma^2$
			\item \emph{Lindeberg's condition:} $\displaystyle \forall_{\varepsilon>0} \sum_{k=1}^{k_n} \E \left( |X_{n,k} - \E X_{n,k}|^2 \1_{\{|X_{n,k} - \E X_{n,k} | > \varepsilon \}} \right) \to 0$
		\end{enumerate}
		Then $S_n \implies \N(m,\sigma^2)$.

	\end{theorem}
	
	\begin{lemma}
		Lindeberg's condition implies
		\begin{itemize}
			\item $\displaystyle \forall_{t>0} \lim_{n \to \infty} \P(\max_{1 \leq k \leq k_n} |X_{n,k}| \geq t) = 0$,
			\item $\displaystyle \lim_{n\to \infty} \max_{1 \leq k \leq k_n} \var(X_{n,k}) \to 0$.
		\end{itemize}

	\end{lemma}
	
	\begin{theorem}[Feller]
		Suppose that $(X_{n,k})$ is a~triangular array, $\displaystyle \sum_{k=1}^{k_n} \E X_{n,k} \to m$, \\ $\displaystyle \sum_{k=1}^{k_n} \var(X_{n,k}) \to \sigma^2$, and $\displaystyle S_n= \sum_{k=1}^{k_n} X_{n,k} \implies \N(m,\sigma^2)$, then if $\displaystyle \max_{1 \leq k \leq k_n} \var(X_{n,k}) \to 0$, then Lideberg's condition is satisfied.
	\end{theorem}





\end{document}
 
 
