\documentclass[12pt]{article}
\usepackage{polski}
\usepackage[utf8]{inputenc}
\usepackage[T1]{fontenc}
\usepackage{amsmath}
\usepackage{amsfonts}
\usepackage{fancyhdr}
\usepackage{lastpage}
\usepackage{multirow}
\usepackage{amssymb}
\usepackage{amsthm}
\usepackage{textcomp}
\usepackage{bbm}
\frenchspacing
\usepackage{fullpage}
\setlength{\headsep}{30pt}
\setlength{\headheight}{12pt}
%\setlength{\voffset}{-30pt}
%\setlength{\textheight}{730pt}
\pagestyle{myheadings}
%\usepackage{kuvio,amscd,diagrams,dcpic,xymatrix,diagxy}
\usepackage{tikz,paralist,mathtools}

\newcommand{\M}{\mathfrak{M}}
\newcommand{\F}{\mathcal{F}}
\newcommand{\G}{\mathcal{G}}
\renewcommand{\P}{\mathbb{P}}
\newcommand{\E}{\mathbb{E}}
\newcommand{\R}{\mathbb{R}}
\newcommand{\B}{\mathcal{B}}
\newcommand{\N}{\mathcal{N}}
\newcommand{\1}{\mathbbm{1}}
\newcommand{\var}{\mathrm{Var}}
\newcommand{\cov}{\mathrm{Cov}}
\newcommand{\corr}{\mathrm{Corr}}
\newcommand{\pn}{\mathrm{p.n.}}
\newcommand{\sgn}{{\mathrm{sgn}}\,}
\newcommand{\bigslant}[2]{{\left.\raisebox{.2em}{$#1$}\middle/\raisebox{-.2em}{$#2$}\right.}}
\newcommand{\mf}[1]{{\mathfrak{#1}}}
\newcommand{\mb}[1]{{\mathbb{#1}}}
\newcommand{\mc}[1]{{\mathcal{#1}}}
\newcommand{\mr}[1]{{\mathrm{#1}}}
\newcommand{\ds}{\displaystyle}
\newcommand{\res}{{\mathrm{res}}}

\newcounter{punkt}

\theoremstyle{plain}
\newtheorem{theorem}[punkt]{Theorem}
\newtheorem{theoremnp}[punkt]{Theorem (no proof)}
\newtheorem{lemma}[punkt]{Lemma}
\newtheorem{proposition}[punkt]{Proposition}

\theoremstyle{definition}
\newtheorem{definition}[punkt]{Definition}
\newtheorem{fact}[punkt]{Fact}
\newtheorem{corollary}[punkt]{Corollary}

\theoremstyle{remark}
\newtheorem{remark}[punkt]{Remark}
\newtheorem{example}[punkt]{Example}
\newtheorem{exercise}[punkt]{Exercise}

\markright{Piotr Suwara\hfill Rachunek prawdopodobieństwa II: 20 listopada 2013 \hfill}

\begin{document}
	\begin{example}
		$(X, Y)$ has density $g_{(X,Y)}(x,y)$.
		Then $$\E(X|Y) = \varphi(y) =
			\frac{\int x g(x,y)\,dx}{\int g(x,y)\,dx}.$$
	\end{example}
	
	\begin{proposition}
		Conditional Expected Value is linear.
	\end{proposition}
	
	\begin{proposition}
		$\E(\E(X | \G)) = \E X$
	\end{proposition}
	
	\begin{proposition}
		$X_1 \geq X_2 \implies \E(X_1 | \G) \geq \E(X_2 | \G)$ almost surely.
	\end{proposition}
	
	\begin{proposition}
		$|\E(X | \G) | \leq \E(|X|\: | \G)$ a.s.,
		in particular CEV is a~contraction in $L^1$
		because $\E| \E(X | \G) | \leq \E |X|$.
	\end{proposition}

	
	\begin{proposition}
		$X$ is $\G$-measurable, then $\E(X|\G) = X$ a.s.
	\end{proposition}
	
	\begin{proposition}
		$X \perp Y \implies \E(X | \G) = \E X$ a.s.
		
		In particular $\E(X | \{ \varnothing, \Omega\}) = \E X$.
	\end{proposition}
	
	\begin{proposition}[Lebesgue monotone convergence thm]
		$0 \leq X_n \nearrow X$, $X_n, X$ integrable,
		then $\E(X_n | \G) \nearrow \E(X|\G)$ a.s.
	\end{proposition}
	
	\begin{proposition}[Lebesgue dominated convergence thm]
		$|X_n| \leq Y$, $\E Y < \infty$, $X_n \to X$ a.s.,
		then $\displaystyle \lim_{n \to \infty} 
				\E(X_n | \G) = \E(X | \G)$ a.s.
	\end{proposition}
	
	\begin{proposition}[Fatou lemma]
		$X_n$ integrable, then $\displaystyle \liminf_{n \to \infty}
		\E\left(X_n | \G\right)
		\geq \E\left(\liminf_{n \to \infty} X_n \middle| \G\right).$
	\end{proposition}
	
	\begin{proposition}
		$X$ is $\G$-measurable, $\E|Y| < \infty, \E|XY| < \infty$,
		then $\E(XY | \G) = X \E(Y | \G)$ a.s.
	\end{proposition}
	
	\begin{proposition}
		$X$ integrable, $\G_1 \subset \G_2$,
		then $\E(X | \G_1) = \E( \E(X | \G_2) | \G_1)
					= \E( \E(X | \G_1) | \G_2)$.
	\end{proposition}
	
	\begin{proposition}[Jensen inequality]
		$\varphi:\R \to \R$ convex, $\E |X| < \infty$,
			$\E |\varphi(X)| < \infty$,
		then $\E(\varphi(X) | \G) \geq \varphi(\E(X | \G))$.
		
		In particular $\E \varphi(X) \geq \varphi(\E X)$.
	\end{proposition}
	
	\begin{corollary}
		$\E |X|^p < \infty$, $p \geq 1$,
		then $|\E ( X | \G )|^p \leq \E\left( |X|^p | \G\right)$.
		
		In particular $\E(\bullet | \G)$ is a~contraction in $L^p$.
	\end{corollary}
	\vspace{0.5cm}
	
	{\bf Filtrations and stopping times}
	
		$T \subset \mb{Z}$
	\begin{definition}[filtration]
		A~\emph{filtration} is a~sequence of $\sigma$-bodies 
		$(\F_t)_{t \in T}$ such that $\F_t \subset \F$ and
		$\F_t \subset \F_s$ for $t < s$.
	\end{definition}
	
	\begin{definition}[stopping time]
		A~\emph{stopping time} with respect to filtration $(\F_t)_{t \in T}$ 
		is a~random variable $\tau: \Omega \to T \cup \{\infty\}$ such that
		$\displaystyle \forall_{t\in T} \{ \tau \leq t\} \in \F_t$.
	\end{definition}
	
	\begin{proposition}
		$\tau: \Omega \to T \cup \{\infty\}$ is a~stopping time
		iff $\forall_{t \in T} \{\tau = t\} \in \F_t$.
	\end{proposition}

\end{document}
 
 
